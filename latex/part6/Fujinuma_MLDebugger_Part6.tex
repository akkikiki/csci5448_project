\documentclass[11pt]{article}
\usepackage[letterpaper,margin=1in]{geometry}
\usepackage{color}
%\usepackage[dvipdfmx]{graphicx}
\usepackage{amsbsy}
\usepackage{amsmath}
\usepackage{amssymb}
\usepackage{physics}
\usepackage{adjustbox}
\usepackage{url}
\usepackage{tabularx}
\usepackage{multicol}
\usepackage{mdwlist}
\usepackage{tablefootnote}
\usepackage{arabtex}
\usepackage{enumitem}
\usepackage{utf8}
\usepackage{tipa}
%\usepackage{floatrow}
\usepackage[font=small,labelfont=bf,tableposition=top]{caption}
\DeclareCaptionLabelFormat{tableonly}{\tablename~\thetable}
%\newfloatcommand{capbtabbox}{table}[][\FBwidth]


\begin{document}
\vspace{-1cm}
\title{\vspace{-2ex} Object Oriented Analysis and Design Project: Part 6\vspace{-2ex}}
%\author{\vspace{-2ex}}
\date{\vspace{-6ex}}
\maketitle

%NOTE: Remember to start from simple and stupid

\section{Summary}
\begin{itemize}[leftmargin=4\parindent,itemsep=-1ex]
 \item Name: Yoshinari Fujinuma
 \item Github link: \url{https://github.com/akkikiki/csci5448_project}
 %\item Title: Visualizing Objective Function
 \item Title: Machine Learning (ML) Model Debugger
 \item Project Summary: A CUI tool that could visualize and interactwith a  machine learning model (e.g., neural networks). The main objective of this tool is to let users save, load, train, and debug a trained model in an feasyc and nituitive way. 
\end{itemize}

\section{Features Implemented and Not Implemented}

\begin{table}[htb]
 \small
 \centering
  \begin{tabular}{|l|l|l|l|}
  \hline
  \bf ID & \bf Requirement                                                          & Implemented \\ \hline
       1 & Plot and visualization of ML models                                      & \\
       2 & Allow to be used by multiple users                                       & \checkmark\\
       3 & Model parameters should be easily tweakable                              & \checkmark\\
       4 & A user cannot load another user's model                                  & \\
       5 & A user can tweak the parameter of a trained model                        & \checkmark\\
       6 & A user can delete a model                                                & \\
       7 & A user cannot delete data                                                & \\
       8 & A user can confirm whether a model's status (finished tranining or not)  & \\
       9 & An admin can delete a user                                               & \\
      10 & An admin can delete a model                                              & \\
      11 & An admin can delete data                                                 & \\
  \hline
  \end{tabular}
 \caption{Project Requirements. The features with \checkmark are implemented in this final project. } 
\end{table}

\section{Design Patterns}

\subsection{Memento}

\subsection{Factory}

\section{What I have Learned}
Classes are more modular than I thought before implementing each design pattern. 
%Stretch Functionality:
%\begin{enumerate}
% \setcounter{enumi}{6}
% \item something
%\end{enumerate}

\end{document}
