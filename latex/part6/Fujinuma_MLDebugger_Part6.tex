\documentclass[11pt]{article}
\usepackage[letterpaper,margin=1in]{geometry}
\usepackage{color}
%\usepackage[dvipdfmx]{graphicx}
\usepackage{amsbsy}
\usepackage{amsmath}
\usepackage{amssymb}
\usepackage{physics}
\usepackage{adjustbox}
\usepackage{url}
\usepackage{tabularx}
\usepackage{multicol}
\usepackage{mdwlist}
\usepackage{tablefootnote}
\usepackage{arabtex}
\usepackage{enumitem}
\usepackage{utf8}
\usepackage{tipa}
%\usepackage{floatrow}
\usepackage[font=small,labelfont=bf,tableposition=top]{caption}
\DeclareCaptionLabelFormat{tableonly}{\tablename~\thetable}
%\newfloatcommand{capbtabbox}{table}[][\FBwidth]


\begin{document}
\vspace{-1cm}
\title{\vspace{-2ex} Object Oriented Analysis and Design Project: Part 6\vspace{-2ex}}
%\author{\vspace{-2ex}}
\date{\vspace{-6ex}}
\maketitle

%NOTE: Remember to start from simple and stupid

\section{Summary}
\begin{itemize}[leftmargin=4\parindent,itemsep=-1ex]
 \item Name: Yoshinari Fujinuma
 \item Github link: \url{https://github.com/akkikiki/csci5448_project}
 %\item Title: Visualizing Objective Function
 \item Title: Machine Learning (ML) Model Debugger
 \item Project Summary: A CUI tool that could visualize and interactwith a  machine learning model (e.g., neural networks). The main objective of this tool is to let users save, load, train, and debug a trained model in an feasyc and nituitive way. 
\end{itemize}

\section{Features Implemented and Not Implemented}

\begin{table}[htb]
 \small
 \centering
  \begin{tabular}{|l|l|l|l|}
  \hline
  \bf ID & \bf Requirement                                                          & Implemented \\ \hline
       1 & Plot and visualization of ML models                                      & \\
       2 & Allow to be used by multiple users                                       & \checkmark\\
       3 & Model parameters should be easily tweakable                              & \checkmark\\
       4 & A user cannot load another user's model                                  & \\
       5 & A user can tweak the parameter of a trained model                        & \checkmark\\
       6 & A user can delete a model                                                & \\
       7 & A user cannot delete data                                                & \\
       8 & A user can confirm model's status  & \checkmark \\
       9 & An admin can delete a user                                               & \\
      10 & An admin can delete a model                                              & \\
      11 & An admin can delete data                                                 & \\
  \hline
  \end{tabular}
 \caption{Project Requirements. The features with \checkmark are implemented in this final project. I prioritized on implementing design patterns. } 
\end{table}

\section{What Changed from the Initial Class Diagram}
{\bf NOTE:} For the class diagram, since I have 20+ classes, I will only show the class diagrams related to the 3 design patterns I implemented in Section~\ref{sec:design}.

\paragraph{More Classes}
Since I did not have specific design pattern in mind when drawing the initial class diagram, I added more while implementing those. 

\paragraph{Separating out intot multiple Models, Views, and Controllers}
Initially, I tried to implement everything into one class, but that soon screw up and violated the ``separation of concerns'' principle. 
Instead, I separate them out and created another class called ``Driver'' to communicate with each controllers. 

\section{Design Patterns}
\label{sec:design}

\subsection{Memento}
Figure~\ref{fig:memento} shows the class diagram for the Memento pattern I implemented. 
I choose Memento pattern to let the users save the current state of tweaked parameters. 
Furthermore, users would like to go back to the state specified rather than doing undos one-by-one. 

I imeplemented this in the ``classifier\_caretaker.py'',  ``classifier\_memento.py'', and ``classifier\_originator.py'' . 

%\section{Class Diagram}
\begin{figure}[htb]
 \centering
     {\includegraphics[width=.75\linewidth]{image/Memento.pdf}}
    \vspace{-2ex}
     \caption{\label{fig:memento}  
        Class diagram for Memento pattern.
     }
\end{figure}

\subsection{Factory}
Figure~\ref{fig:factory} shows the class diagram for the Factory pattern I implemented. 
I choose Factory pattern because I wanted the same function across all corpus file types. 
Therefore, I thought factory design pattern is most suitable for it. 

I imeplemented this pattern in ``corpus.py''.

\begin{figure}[htb]
 \centering
     {\includegraphics[width=.75\linewidth]{image/Factory.pdf}}
    \vspace{-2ex}
     \caption{\label{fig:factory}  
        Class diagram for Factory pattern.
     }
\end{figure}

\subsection{State}
Figure~\ref{fig:state} shows the class diagram for the State pattern I implemented. 
I choose State pattern to implement the ``Undo'' function to go back one menu transition. 
In the future, there will be more than one level hierarchy in the menu transition. 

I imeplemented this pattern in ``menu.py''.

\begin{figure}[htb]
 \centering
     {\includegraphics[width=.75\linewidth]{image/State.pdf}}
    \vspace{-2ex}
     \caption{\label{fig:state}  
        Class diagram for State pattern.
     }
\end{figure}


\section{What I have Learned}
I learned few things: (1) Initial planning by the class diagrams help brainstorm the software, but that will certainly change as we imoplment them, and (2) classes are more modular than I initially thought before implementing and learning each design pattern. 
%Stretch Functionality:
%\begin{enumerate}
% \setcounter{enumi}{6}
% \item something
%\end{enumerate}

\end{document}
