\documentclass[11pt]{article}
\usepackage[letterpaper,margin=1in]{geometry}
\usepackage{color}
%\usepackage[dvipdfmx]{graphicx}
\usepackage{amsbsy}
\usepackage{amsmath}
\usepackage{amssymb}
\usepackage{physics}
\usepackage{adjustbox}
\usepackage{url}
\usepackage{tabularx}
\usepackage{multicol}
\usepackage{mdwlist}
\usepackage{tablefootnote}
\usepackage{arabtex}
\usepackage{enumitem}
\usepackage{utf8}
\usepackage{tipa}
%\usepackage{floatrow}
\usepackage[font=small,labelfont=bf,tableposition=top]{caption}
\DeclareCaptionLabelFormat{tableonly}{\tablename~\thetable}
%\newfloatcommand{capbtabbox}{table}[][\FBwidth]


\begin{document}
\vspace{-1cm}
\title{\vspace{-2ex} Object Oriented Analysis and Design Project: Part 2\vspace{-2ex}}
%\author{\vspace{-2ex}}
\date{\vspace{-6ex}}
\maketitle

%NOTE: Remember to start from simple and stupid

\begin{itemize}[leftmargin=4\parindent,itemsep=-1ex]
 \item Name: Yoshinari Fujinuma
 \item Github link: \url{https://github.com/akkikiki/csci5448_project}
 %\item Title: Visualizing Objective Function
 \item Title: ML Debugger
 \item Project Summary: A CUI tool that could visualize the shape of an objective function of a given machine learning model (e.g., neural networks). The main objective of this tool is to let users debug while training a model. 
\end{itemize}

%Functionality:
%\begin{enumerate}[leftmargin=4\parindent,itemsep=-1ex]
% \item Draw the current objective function on a 3D plot
% \item Users look into the model (n-dimentional tensors)
% \item Users terminate the learning
% \item Users resume the learning
% \item Users can visualize the objective funcation of the previously trained model
% \item Users look into how the learning of the model is performed
% \item Users can save the trained models
% \item Users can load the trained models
% \item Users can save the plots
% \item Only an admin can delete the trained models
%\end{enumerate}

\section{Project Requirements}
%\begin{enumerate}[leftmargin=4\parindent,itemsep=-1ex]
% \item A plot
% \item high-dimensional to 3-dimensional
% \item intermediate representation to 3-dim.
%\end{enumerate}

\begin{table}[htb]
 \small
 \centering
  \begin{tabular}{|l|l|l|l|}
  \hline
  \bf ID & \bf Requirement                                                           \\ \hline
       1 & Plot and visualization of ML models                                       \\
       2 & Allow to be used by multiple users                                        \\
       3 & Model parameters should be easily tweakable                               \\
       4 & A user cannot load another user's model                                   \\
       5 & A user can tweak the parameter of a trained model                         \\
       6 & A user can delete a model                                                 \\
       7 & A user cannot delete data                                                 \\
       8 & A user can confirm whether a model's status (finished tranining or not)   \\
       9 & An admin can delete a user                                                \\
      10 & An admin can delete a model                                               \\
      11 & An admin can delete data                                                  \\
  \hline
  \end{tabular}
 \caption{Project Requirements} 
\end{table}



%\section{User Requirements}
%\begin{enumerate}[leftmargin=4\parindent,itemsep=-1ex]
% \item A user cannot load another user's model
% \item A user can tweak the parameter of a trained model
% \item A user can delete a model
% \item A user cannot delete data
% \item A user can confirm whether a model's status (finished tranining or not)
% \item An admin can delete a user
% \item An admin can delete a model
% \item An admin can delete data
%\end{enumerate}
%
%\section{Funcationality Requirements}
%\begin{enumerate}[leftmargin=4\parindent,itemsep=-1ex]
% \item Keep track of saved models by individual user
%\end{enumerate}

\newpage
\section{UI Mockups}
Menu/user interaction

\begin{figure}[htb]
 \centering
     {\includegraphics[width=.7\linewidth]{image/ui_mockup.pdf}}
    \vspace{-2ex}
     \caption{\label{fig:ui_mock}  
       UI Mockup 
     }
\end{figure}

\newpage
\section{Class Diagram}
\begin{figure}[htb]
 \centering
     {\includegraphics[width=.9\linewidth]{image/class_diagram.pdf}}
    \vspace{-2ex}
     \caption{\label{fig:class_diagram}  
        Class diagram
     }
\end{figure}



%Stretch Functionality:
%\begin{enumerate}
% \setcounter{enumi}{6}
% \item something
%\end{enumerate}

\end{document}
